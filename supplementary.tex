\documentclass[a4paper]{article}

\usepackage{hyperref}
\usepackage{amsmath}
\usepackage{amssymb}
\usepackage{bm}


%
% Some new commands I use in this text
%
\newcommand{\OpSphere}{\bm{\mathcal{S}}}
\newcommand{\OpRot}{\bm{\mathcal{R}}}
\newcommand{\OpDistance}{\bm{\mathcal{D}}}
\newcommand{\OpCumsum}{\bm{\mathcal{C}}}
\newcommand{\OpInt}{\bm{\mathcal{I}}}
\newcommand{\OpCamera}{\bm{\mathcal{P}}}
\newcommand{\OpDiag}[1]{\mathrm{diag}\left\{#1\right\}}
\newcommand{\Grad}[1]{\bm{\triangledown_{#1}}}
\newcommand{\argmin}{\mathrm{arg}\min}
\newcommand{\curly}[1]{\left\{#1\right\}}
\newcommand{\roundy}[1]{\left(#1\right)}
\newcommand{\recty}[1]{\left[#1\right]}
\newcommand{\PartDeriv}[2]{\frac{\partial{#1}}{\partial{#2}}}
\newcommand{\vect}[1]{\bm{#1}}
\newcommand{\mat}[1]{\bm{#1}}
\newcommand{\transpose}[1]{{#1}^\intercal}
\newcommand{\derivsym}[1]{\,d{#1}}

\begin{document}

\title{\vspace{-2cm}Supplementary material: \\Sky Tomography for 3D Aerosol Distribution Recovery}
\author{Paper ID: 19}
\date{}

\maketitle

\section{Gradient Derivation}
\label{sec:gradient-derivation}

Here we detail the derivation of the gradient of the objective
function:
\begin{equation}
  \label{eq:objectiveA}
  F({\bf n})
  = \sum_{c=1}^{N_{\rm views}}
  \| {\bf i}^{\rm measured}_c - {\bf i}_c({\bf n})\|^2_2
\end{equation}
Here ${\bf i}^{\rm measured}_c$ is the image captured by camera $c$
and ${\bf i}_c({\bf n})$ is an image modeled from the viewpoint of
camera $c$ according to the density distribution ${\bf n}$. In the
article we show that ${\bf i}_c({\bf n})$ can be expressed by
\begin{align}
  {\bf i}_c= L^{\rm TOA} {\vect{\Pi}}_c \Big\{ e^{-(\vect{\tau}_c^{\rm
      air} + {\sigma}^{\rm aerosol} {\bf D}_c {\bf n}) }\odot
  \;\;\;\;\;\;\;\;\;\;\;\;\;\;\;\;\;\;\;\;\;
  \nonumber \\
  [\tilde{\vect{\alpha}}^{\rm air}_c + %\right.
  \varpi^{\rm aerosol}\sigma^{\rm aerosol} P^{\rm
    aerosol}_g(\vect{\Phi}^{\rm scatter}_c) \odot{\bf n} ] \Big\} .
  \label{eq:bigIA}
\end{align}

The gradient of $F$ with respect to ${\bf n}$ is
\begin{align}
  \Grad{{\bf n}} F = -2\sum_{c=1}^{N_{\rm views}}
  \transpose{\left[{\bf J}_{{\bf i}_c}({\bf n})\right]} [{\bf i}^{\rm
    measured}_c - {\bf i}_c({\bf n})] \;.
  \label{eq:gradient1}
\end{align}
Here the matrix ${\bf J}_{{\bf i}_c}({\bf n})$ is the Jacobian of the
vector ${\bf i}_c$ with respect to ${\bf n}$. Element $(\Theta,k)$ of
this matrix differentiates the intensity in pixel ${\bf \Theta}$ (in
viewpoint $c$) with variation of the density at voxel $k$, i.e.,
$\partial i_c({\bf \Theta})/\partial{n^{\rm aerosol}(k)}$.

In order to describe the derivation of the Jacobian ${\bf J}_{{\bf i}_c}({\bf n})$, we first show some results relating to differentiation. Let $\vect{n}$
be a vector of length $q$. Let $\vect{g}(\vect{n})$ be a vector
function: it outputs a vector of length $r$. Let $\mat{B}$ be a $q
\times r$ matrix and
\begin{align}
  \vect{g}(\vect{n}) = \exp^{-\transpose{\mat{B}}\vect{n}}.
  \label{eq:example1}
\end{align}
In Eq.~(\ref{eq:example1}), the exponential is element-wise (not
raising an operator to some power). Then
\begin{align}
  \label{eq:partial2}
  \PartDeriv{\vect{g}}{\vect{n}} &= - \mat{B} \,
  \OpDiag{\exp^{-\transpose{\mat{B}}\vect{n}}}
\end{align}
Here we denote by $\OpDiag{\vect{v}}$ conversion of vector $\vect{v}$
into a diagonal matrix, whose main diagonal elements correspond to the
elements of $\vect{v}$. Using~(\ref{eq:partial2}) we can calculate
\begin{align}
  \label{eq:partial4}
  \PartDeriv{\exp^{-(\vect{\tau}_c^{\rm air} + {\sigma}^{\rm aerosol}
      {\bf D}_c {\bf n})}}
  {\vect{n}} &= - {\sigma}^{\rm aerosol}\transpose{\OpDistance_c} \,
  \OpDiag{\exp^{-(\vect{\tau}_c^{\rm air} + {\sigma}^{\rm aerosol}
      {\bf D}_c {\bf n})}}
\end{align}
In~(\ref{eq:partial4}) we assume that $\vect{\tau}_c^{\rm air}$ is
known and constant.

To calculate the derivative of the Hadamard product
in~(\ref{eq:bigIA}) we first note the following: Let
$\vect{u}(\vect{n})$ be a vector function that outputs a vector of
length $r$. Let $\mat{B}$ be a $r \times q$ matrix. Then,
\begin{align}
  \label{eq:partial1}
  \PartDeriv{\transpose{\mat{B}} (\vect{g} \odot \vect{u})}{\vect{n}}
  = \left[ \PartDeriv{\vect{g}}{\vect{n}} \OpDiag{\vect{u}} +
    \PartDeriv{\vect{u}}{\vect{n}} \OpDiag{\vect{g}} \right] \mat{B}.
\end{align}
Then,
\begin{align}
  \label{eq:partial3}
  \PartDeriv{\, \varpi^{\rm aerosol}\sigma^{\rm aerosol} P^{\rm
      aerosol}_g(\vect{\Phi}^{\rm scatter}_c) \odot{\bf n}}{\vect{n}}
  =
  \;\;\;\;\;\;\;\;\;\;\;\;\;\;\;\;\;\;\;\;\;\;\;\;\;\;\;\;\;\;\;\;\;\;\;\;\;\;\;\;\;\;
  \nonumber \\
  \varpi^{\rm aerosol}\sigma^{\rm aerosol}\OpDiag{P^{\rm
      aerosol}_g(\vect{\Phi}^{\rm scatter}_c)}
  \;\;\;\;\;\;\;\;\;\;\;\;\;\;\;\;\;\;\;\;
\end{align}

Based on~(\ref{eq:partial4},~\ref{eq:partial1})
and~(\ref{eq:partial3}) we can derive a close-form expression for the
Jacobian of~(\ref{eq:bigIA}):
\begin{align}
  {\bf J}_{{\bf i}_c}({\bf n}) &= L^{\rm TOA}({\bf A}-{\bf B}){\bf C}
  ,
  \label{eq:gradient2}
\end{align}
where
\begin{align}
  \label{eq:gradient3}
  {\bf A} &= \varpi^{\rm aerosol}\sigma^{\rm aerosol}
  \OpDiag{ P_g^{\rm aerosol}(\vect{\Phi}^{\rm scatter}_c)} \nonumber\\
  {\bf B} &= {\sigma}^{\rm aerosol}\transpose{{\bf D}_c} \nonumber\\
  & \;\;\; \OpDiag{[\tilde{\vect{\alpha}}^{\rm air}_c + \varpi^{\rm
      aerosol}\sigma^{\rm aerosol} P^{\rm aerosol}_g(\vect{\Phi}^{\rm
      scatter}_c) \odot{\bf n}
    ]} \nonumber \\
  {\bf C} &= \OpDiag{\exp[-(\vect{\tau}_c^{\rm air} + {\sigma}^{\rm
      aerosol} {\bf D}_c {\bf n})]} \transpose{{\vect{\Pi}}_c}
\end{align}

\end{document}

