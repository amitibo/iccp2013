\documentclass[a4paper]{article}

\usepackage{hyperref}
\usepackage{amsmath}
\usepackage{amssymb}
\usepackage{bm}


%
% Some new commands I use in this text
%
\newcommand{\OpSphere}{\bm{\mathcal{S}}}
\newcommand{\OpRot}{\bm{\mathcal{R}}}
\newcommand{\OpDistance}{\bm{\mathcal{D}}}
\newcommand{\OpCumsum}{\bm{\mathcal{C}}}
\newcommand{\OpInt}{\bm{\mathcal{I}}}
\newcommand{\OpCamera}{\bm{\mathcal{P}}}
\newcommand{\OpDiag}[1]{\mathrm{diag}\left\{#1\right\}}
\newcommand{\Grad}[1]{\bm{\triangledown_{#1}}}
\newcommand{\argmin}{\mathrm{arg}\min}
\newcommand{\curly}[1]{\left\{#1\right\}}
\newcommand{\roundy}[1]{\left(#1\right)}
\newcommand{\recty}[1]{\left[#1\right]}
\newcommand{\PartDeriv}[2]{\frac{\partial{#1}}{\partial{#2}}}
\newcommand{\vect}[1]{\bm{#1}}
\newcommand{\mat}[1]{\bm{#1}}
\newcommand{\transpose}[1]{{#1}^\intercal}
\newcommand{\derivsym}[1]{\,d{#1}}

\begin{document}

\title{\vspace{-2cm}Supplementary material: \\Sky Tomography for 3D Aerosol Distribution Recovery}
\author{Paper ID: 19}
\date{}

\maketitle

\section{Gradient Derivation}
\label{sec:gradient-derivation}

Here we detail the derivation of~(\ref{eq:gradient2}).  We use
matrices to implement the different operators. Using matrices
simplifies the calculation of derivatives and the use of optimization
tools. Note that~(\ref{eq:bigIA}) in column stack notation is
\begin{align}
  \vect{i}_c=(\rm{NA})^2\mathrm{L}^{\rm TOA} \OpCamera \, \OpInt \,
  \roundy{ \roundy {\OpSphere_c \, \roundy{\vect{\alpha}^{\rm
          aerosol}_c + \vect{\alpha}^{\rm air}_c} } \odot
    \exp^{-(\vect{\tau}^{\rm aerosol}_c + \vect{\tau}^{\rm air}_c)}}.
  \label{eq:atsensor-matrix}
\end{align}
First we show some results relating to differentiation. Let $\vect{a}$
be a vector of length $q$. Let $\vect{g}(\vect{a})$ be a vector
function: it outputs a vector of length $r$. Let $\mat{B}$ be a $q
\times r$ matrix and
\begin{align}
  \vect{g}(\vect{a}) = \exp^{-\transpose{\mat{B}}\vect{a}}.
  \label{eq:example1}
\end{align}
In Eq.~(\ref{eq:example1}), the exponential is element-wise (not
raising an operator to some power). Then
\begin{align}
  \label{eq:partial2}
  \PartDeriv{\vect{g}}{\vect{a}} &= - \mat{B} \,
  \OpDiag{\exp^{-\transpose{\mat{B}}\vect{a}}}
\end{align}
Using~(\ref{eq:partial2}) and~(\ref{eq:tau})
\begin{align}
  \label{eq:partial4}
  \PartDeriv{\exp^{-(\vect{\tau}^{\rm aerosol}_c + \vect{\tau}^{\rm
        air}_c)}}{\vect{a}} &= - \transpose{\OpDistance_c} \,
  \OpDiag{\exp^{-(\vect{\tau}^{\rm aerosol}_c + \vect{\tau}^{\rm
        air}_c)}}
\end{align}

To calculate the derivative of the Hadamard product
in~(\ref{eq:atsensor-matrix}) we first note the following: Let
$\vect{u}(\vect{a})$ be a vector function that outputs a vector of
length $r$. Let $\mat{B}$ be a $r \times q$ matrix. Then,
\begin{align}
  \label{eq:partial1}
  \PartDeriv{\transpose{\mat{B}} (\vect{g} \odot \vect{u})}{\vect{a}}
  = \left[ \PartDeriv{\vect{g}}{\vect{a}} \OpDiag{\vect{u}} +
    \PartDeriv{\vect{u}}{\vect{a}} \OpDiag{\vect{g}} \right]
  \mat{B}.
\end{align}
%
% TODO: Understand if Af \odot Ag = A(f \odot g) and in what terms (on
% A) this is important as in the docs I use the left case and in the
% code I use the right case. It seems to me that it is correct but I
% don't have a proof.
%

Suppose the air distribution is known. Therefore, $\vect{\alpha}^{\rm
  air}_c$ and $\vect{\tau}^{\rm air}_c$ are constant vectors. Then,
\begin{align}
  \label{eq:partial3}
  \PartDeriv{\OpSphere_c \, \vect{\alpha}^{\rm aerosol}_c}{\vect{a}}
  &= \varpi \, \sigma^{\rm aerosol} \, \OpDiag{P^{\rm
      aerosol}(\vect{\mu_c})} \, \transpose{\OpSphere_c}
\end{align}

Based on~(\ref{eq:partial4},~\ref{eq:partial1})
and~(\ref{eq:partial3}) we can calculate the gradient
of~(\ref{eq:atsensor-matrix}):
\begin{align}
  \PartDeriv{\vect{i}_{\rm c}}{\vect{a}} &=
  \mat{J}_{\hat{\vect{i}}_c}(\vect{a})= \nonumber \\
  &= (\rm{NA})^2 \mathrm{L}^{\rm TOA} \, \biggl( \varpi \, \sigma^{\rm
    aerosol} \, \OpDiag{P^{\rm aerosol}(\vect{\mu}_c)} \,
  \transpose{\OpSphere}_c \OpDiag{\exp^{-(\vect{\tau}^{\rm aerosol}_c
      + \vect{\tau}^{\rm
        air}_c)}} \nonumber \\
  &\quad - \transpose{\OpDistance_c} \, \OpDiag{\exp^{-(\vect{\tau}^{\rm
        aerosol}_c + \vect{\tau}^{\rm air}_c)}}\OpDiag{\OpSphere_c \,
    (\vect{\alpha}^{\rm aerosol}_c + \vect{\alpha}^{\rm air}_c)}
  \biggr)
  \, \transpose{\OpInt} \, \transpose{\OpCamera} \nonumber \\
  &= (\rm{NA})^2\mathrm{L}^{\rm TOA} \, \biggl( \sigma^{\rm aerosol}
  \, \varpi \,
  \OpDiag{P^{\rm aerosol}(\vect{\mu}_c)} \, \transpose{\OpSphere_c} - \nonumber \\
  &\quad \transpose{\OpDistance_c} \, \OpDiag{\OpSphere_c \,
    (\vect{\alpha}^{\rm aerosol}_c + \vect{\alpha}^{\rm air}_c )}
  \biggr) \OpDiag{\exp^{-(\vect{\tau}^{\rm aerosol}_c +
      \vect{\tau}^{\rm air}_c)}} \, \transpose{\OpInt} \,
  \transpose{\OpCamera}
\end{align}

\end{document}
